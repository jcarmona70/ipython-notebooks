%\documentclass[11pt,heading,asymmetric,exercises]{uniexer}
\documentclass[11pt,heading,asymmetric,exercises,withsolutions]{uniexer}

\usepackage{a4wide}
\usepackage[T1]{fontenc}
\usepackage[spanish]{babel}
\usepackage[utf8]{inputenc}
\usepackage{pstricks}
\usepackage{pst-node}
\usepackage{pst-plot}
\usepackage{graphicx}
\usepackage{amssymb}
\usepackage{marvosym}
\usepackage[colorlinks]{hyperref}
\usepackage{codex}
\usepackage{listings}
\selectlanguage{spanish}  


% $Id: datosAsignatura.tex,v 1.2 2011-11-11 19:31:47 luis Exp $
\exercita{defs}{language-logo={Python}}
\exercita{defs}{language={Python}}
\exercita{defs}{subject={Inform\'atica},%
                academic-year={2013},%
                group={A,B,C,D,E},%
                institution={Facultad de CC. Matem\'aticas},
                degree={Grados en Matem\'aticas}}           
\exercita{defs}{doc-number={1},history?,doc-titles?,doc-title={Expresiones y Entrada/Salida},bibliography?,doc-hint-style=hint}




\exercita{defs}{doc-number={1 (Soluciones)},doc-titles?,dos-his?,dos-h=auto,bibliography?,doc-title={Expresiones}}           
\setlength{\textheight}{24cm}
\setlength{\headheight}{0em}
\begin{document}


\exercita{base}{programacion}

\vspace*{1em}

%Para otro curso: /home/carlos/docencia/exercita-db/programacion/programacion/entrenamiento.formacionRectangular y formacionTriangular.

\exercita{load}{matematicas.analisis.seriesYsucesiones.terminogeneral}[*,solMatematica]
%\exercita{load}{matematicas.logica.diadicas}[*,solImperativa] 
%\exercita{load}{matematicas.geometria.cortesCirculo{label=*}}[*,solImperativa]
\exercita{load}{matematicas.geometria.circulosTangentes}[*,solImperativa{en-funcion}] 
%\exercita{load}{cotidianos.congruencia-Zeller{expresion}}[*,solPascal]
\exercita{load}{recreativos.siete-mensajeros{label=*}}[sol]
%\exercita{load}{recreativos.voltereta{label=*}}[*,solPascal]
\exercita{load}{cotidianos.calculosDepositosHipotecas}[*,solPythonCastellano]
\exercita{load}{matematicas.aritmetica.cuad-perf}[*,solImperativa{en-funcion}]
\exercita{load}{matematicas.geometria.triangulo-expresiones}[*,solPython]
\exercita{load}{matematicas.geometria.rectangulo-expresiones{label=*}}[*,solPython]
\exercita{load}{matematicas.geometria.rectangulo-paralelo-expresiones}[*,solPython]
\exercita{load}{matematicas.analisis.comparacionDeReales}[*,solImperativa]

\bibliographystyle{alpha}

\end{document}
 
